\documentclass[a4paper,12pt]{book}

\usepackage[utf8]{inputenc}
\usepackage[T1]{polski}
\usepackage{helvet}
\usepackage{indentfirst}
\usepackage{graphicx}
\usepackage{color}
\usepackage{geometry}
\usepackage{sectsty}
\usepackage{verbatim}
\usepackage{amsmath}



\geometry{hmargin={2cm, 2cm}, height=10.0in}

%\renewcommand{\section}[1]{
%	\noindent\rule{\textwidth}{0.2pt}\vspace{-0.5\parindent}
%	\noindent\begin{center} {\fontsize{14}{17}\bf #1}\end{center}\vspace{-\parindent}
%	\noindent\rule{\textwidth}{0.4pt}
%}


\begin{document}

% =====  STRONA TYTUŁOWA PRACY INŻYNIERSKIEJ ====

\thispagestyle{empty}
%% ------------------------ NAGŁÓWEK STRONY ---------------------------------
\includegraphics[height=37.5mm]{fig/logo_kolor.eps}\\
\rule{30mm}{0pt}
{\large\textsf{Wydział Fizyki i Informatyki Stosowanej}}\\
\rule{\textwidth}{3pt}\\
\rule[2ex]
{\textwidth}{1pt}\\
\vspace{7ex}
\begin{center}
{\bf\LARGE\textsf{Praca inżynierska}}\\
\vspace{13ex}
% --------------------------- IMIE I NAZWISKO -------------------------------
{\bf\Large\textsf{Rafał Szęszoł}}\\
\vspace{3ex}
{\sf \small kierunek studiów:} {\bf\small\textsf{Informatyka Stosowana}}\\
\vspace{15ex}
%% ------------------------ DANE PRACY --------------------------------------
{\bf\huge\textsf{Wykorzystanie równania adwekcji w symulacji mieszania kolorów.}}\\
\vspace{14ex}
{\sf \Large Opiekun:} {\bf\Large\textsf{dr hab. inż. Tomasz Chwiej}}\\
\vspace{22ex}
\textsf{\bf\large\textsf{Kraków, styczeń 2017}}
\end{center}
%% =====  STRONA TYTUŁOWA PRACY INŻYNIERSKIEJ  ====

\newpage

%% =====  TYŁ STRONY TYTUŁOWEJ PRACY INŻYNIERSKIEJ  ====
{\sf Oświadczam, świadomy odpowiedzialności karnej za poświadczenie nieprawdy,
że niniejszą pracę dyplomową wykonałem osobiście i samodzielnie i nie korzystałem
ze źródeł innych niż wymienione w pracy.}

\vspace{14ex}

\begin{flushright}
................................................................. \\
{\sf (czytelny podpis)}
\end{flushright}
%% =====  TYL STRONY TYTUŁOWEJ PRACY INŻYNIERSKIEJ  ====

\newpage
\linespread{1.3}
\selectfont

\noindent
Na kolejnych dwóch stronach proszę dołączyć kolejno recenzje pracy popełnione przez Opiekuna oraz Recenzenta
(wydrukowane z systemu MISIO i podpisane przez odpowiednio Opiekuna i Recenzenta pracy).
Papierową wersję pracy (zawierającą podpisane recenzje) proszę złożyć w dziekanacie celem rejestracji.

\vspace{85mm}

\newpage
\tableofcontents
\chapter{Wstęp}
\section{Problem szybkiego mieszania}
\chapter{Wprowadzenie teoretyczne}
\section{Adwekcja}
\section{Równanie Naviera Stokesa}
%%%%%%%%%%%%%%%%%%%%%%%%%%%%%%%%%%%%%%%%%%%%%%%%%%%%%%%%%%%%%%%%%%%%%%%%%%%%%%%%%%%%%%%%%%%%%%%%%%%%
\chapter{Równanie adwekcji metodą Cranka-Nicholson}
\section{Crank-Nicholson}
\section{Analiza Von Neumanna}
\section{Bicgstab}
\section{Wyniki z ujemną gęstością}
%%%%%%%%%%%%%%%%%%%%%%%%%%%%%%%%%%%%%%%%%%%%%%%%%%%%%%%%%%%%%%%%%%%%%%%%%%%%%%%%%%%%%%%%%%%%%%%%%%%%
\chapter{Problem dyfuzji numerycznej}
\section{Analiza problemu}
\section{FCT}

%%%%%%%%%%%%%%%%%%%%%%%%%%%%%%%%%%%%%%%%%%%%%%%%%%%%%%%%%%%%%%%%%%%%%%%%%%%%%%%%%%%%%%%%%%%%%%%%%%%%
\chapter{Schemat square-root}
\section{Omówienie metody}
Metoda jako lekarstwo.
\section{Wady metody square-root}
\section{FCT}
Boris Brook
Zalesak
%%%%%%%%%%%%%%%%%%%%%%%%%%%%%%%%%%%%%%%%%%%%%%%%%%%%%%%%%%%%%%%%%%%%%%%%%%%%%%%%%%%%%%%%%%%%%%%%%%%%
\chapter{Problem prędkości na brzegach}
\section{Modelowanie funkcji wirowości}
\section{Niezerowa prędkość na brzegach}
Niezerowa prędkość na brzegach powoduje znikanie pakietu.
%%%%%%%%%%%%%%%%%%%%%%%%%%%%%%%%%%%%%%%%%%%%%%%%%%%%%%%%%%%%%%%%%%%%%%%%%%%%%%%%%%%%%%%%%%%%%%%%%%%%
\chapter{Statyczne mieszanie kolorów}
\section{Wprowadzenie}
\section{Mieszanie}
\section{Stała mieszania}
Badanie homogeniczności mieszaniny
\section{Schematy mieszania}
%%%%%%%%%%%%%%%%%%%%%%%%%%%%%%%%%%%%%%%%%%%%%%%%%%%%%%%%%%%%%%%%%%%%%%%%%%%%%%%%%%%%%%%%%%%%%%%%%%%%
\chapter{Dynamiczne mieszanie kolorów}
\section{Wprowadzenie}
\section{Mieszanie}
\section{Stała mieszania}
%%%%%%%%%%%%%%%%%%%%%%%%%%%%%%%%%%%%%%%%%%%%%%%%%%%%%%%%%%%%%%%%%%%%%%%%%%%%%%%%%%%%%%%%%%%%%%%%%%%%
\chapter{Bibliografia}
Wzory do wyprowadzające równanie różnicowe, Crank-Nicholson.
Równanie adwekcji:
\begin{gather}
%%%%%%%%%%%%%%%%%%%%%%%%%%%%%%%%%%%%%%%%%%%%%%%%%%%%%%%%%%%%%%%%%%%%%%%%%%%%%%%%%%%%%%%%%%%%%%%%%%%%
\frac{\partial u}{\partial t} + \nabla(uV) = 0 \\ 
%after differential
\frac{u^{n+1} - u^n}{\Delta t} = - \nabla\left[\frac{u^{n+1}V^{n+1} + u^n V^n}{2}\right] \\
%extract 2 before braces
\frac{u^{n+1} - u^n}{\Delta t} = - \frac{1}{2}\left[{u^{n+1}V^{n+1} + u^n V^n}\right] \\
%times \Delta t, move u^n on the right side
u^{n+1} = u^n - \frac{\Delta t}{2}\left[{\nabla u^{n+1}V^{n+1} + \nabla u^nV^n}\right] \\
%n+1 on the left and n on the right
u^{n+1} + \frac{\Delta t}{2}{\nabla u^{n+1}V^{n+1}}=u^n - \frac{\Delta t}{2}{\nabla u^n V^n}\\
%applyin nabla on u^{n+1} and u^n
\begin{split}
u_{ij}^{n+1} + \frac{\Delta t}{2}\left[V_x^{n+1} \frac{u_{i+1j}^{n+1} - 2u_{ij}^{n+1} + u_{i-1j}^{n+1}}{{\Delta x}^2}
	+ V_y^{n+1} \frac{u_{ij+1}^{n+1} - 2u_{ij}^{n+1} + u_{ij-1}^{n+1}}{{\Delta y}^2}\right] \\
= u_{ij}^{n} - \frac{\Delta t}{2}\left[V_x^{n} \frac{u_{i+1j}^{n} - 2u_{ij}^{n} + u_{i-1j}^{n}}{{\Delta x}^2}
	+ V_y^{n} \frac{u_{ij+1}^{n} - 2u_{ij}^{n} + u_{ij-1}^{n}}{{\Delta y}^2}\right] 
\end{split} \\ 
%po przegrupowaniu wyrazow
\begin{split}
\alpha^{n+1} u_{ij}^{n+1} + \beta^{n+1} u_{i+1j}^{n+1} + \beta^{n+1} u_{i-1j}^{n+1}
	+ \gamma^{n+1} u_{ij+1}^{n+1} + \gamma^{n+1} u_{ij-1}^{n+1}\\
= \alpha^{n} u_{ij}^{n} - \beta^{n} u_{i+1j}^{n} - \beta^{n} u_{i-1j}^{n}
	- \gamma^{n} u_{ij+1}^{n} - \gamma^{n} u_{ij-1}^{n} 
\end{split}\\
%wspolczynniki
\alpha^{n+1} = 1 - \frac{\Delta t V_{x}^{n+1}}{\Delta t x^2} - \frac{\Delta t V_{y}^{n+1}}{\Delta t y^2},
\alpha^{n} = 1 - \frac{\Delta t V_{x}^{n}}{\Delta t x^2} - \frac{\Delta t V_{y}^{n}}{\Delta t y^2}\\
\beta^{n+1} = \frac{\Delta t V_{x}^{n+1}}{\Delta t x^2},
\beta^{n} = - \frac{\Delta t V_{x}^{n}}{\Delta t x^2}\\
\gamma^{n+1} = \frac{\Delta t V_{y}^{n+1}}{\Delta t y^2},
\gamma^{n} = - \frac{\Delta t V_{y}^{n}}{\Delta t y^2}\\
\end{gather}
\end{document}
